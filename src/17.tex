\newpage
\section{Az utazóügynök probléma}

Adott egy $G$ teljes gráf és egy $c:V \mapsto \mathbb{R}^+$ él súlyozás,
keressünk minimális összsúlyú Hamilton--kört. A probléma NP--nehéz. Az általános
utazóügynök problémára nem létezik $k$--approximációs algoritmus (feltéve, hogy
P$\ne$NP), mivel a probléma visszavezethető egy nem teljes gráfban Hamilton--kör
keresésére.

Legyen egy $n$--pontú $G$ (nem teljes) gráf, a bemenet, hozzá tartozó $G'$ pedig
teljes gráf. Ha $G$-ben szomszédos két pont, akkor $G$-ben egységnyi az élsúly,
egyébként $kn$. Ha $G$-ben van Hamilton-kör, akkor $G'$-ben a minimális
összsúlyú Hamilton-kör összsúlya $n$, egyébként legalább $n-1+kn$. 

Mivel ez több, mint az optimum $k$--szorosa, ezért nincs ilyen közelítés.
Egyébként az algoritmus meg tudná különböztetni a Hamilton-körrel rendelkező és
nem rendelkező gráfokat. Ugyanez $kn$ helyett tetszőleges $f(n)$ függvénnyel,
amely polinom időben kiszámolható, is igaz marad. Ezért a bemenettől függő
approximáció sem létezik.

\subsection{Metrikus utazó ügynök}

\emph{A problémát egészítsük ki azzal a kikötéssel, hogy a háromszög
egyenlőtlenség a pontok közti élre teljesül (csúcs $\mapsto$ sík pontjai, élek
súlya $\mapsto$ pontok közti távolság). Erre már létezik approximációs
algoritmus.}

\subsubsection{$2$-approximációs algoritmus}

Ha egy $n$ pontú teljes gráfban ($G$):
\begin{enumerate}
  \item vesszünk egy minimális $F$ feszítőfát\footnote{\emph{Kruskall algoritmusa} -- 
rendezzük az éleket növekvő sorrendbe, majd csökkenő sorrendbe vegyünk be $n-1$ élet
amely a gráfban nem hoz létre kört.},
  \item duplázzuk meg $F$ éleit, majd keresünk egy Euler--kört,
  \item az ismétlődő pontokat a körből vágjuk le, elkészítve egy Hamilton-kört.
\end{enumerate}

Ha létezik a Hamilton--kör legyen ennek költsége OPT. Ebből, ha elhagyunk egy
élet, kapunk egy utat amely egy Hamilton--út, értéke kisebb mint OPT és
feszitőfa is. Az algoritmus első lépésében megkaptuk az optimális feszitőfát,
amely e OPT--nál biztos kisebb (vagy egyenlő).

Ennek megduplázásával $2 \cdot$ OPT költségű zárt élsorozatott kaptunk, amely
hosszát az elhagyások csak csökkenteni tudják (a háromszög feltétel miatt),
tehát az algoritmus által talált összköltség $\leq 2 \cdot$OPT.

\begin{align*}
 c(\mbox{talált Hamilton-kör}) &\leq c(\mbox{Euler--kör})		         \\ 
 						       &= 2  \cdot c(\mbox{minimális feszítőfa})       \\
 							   & \leq 2 \cdot c(\mbox{minimális Hamilton-út}) \\ 
 							   & < 2\cdot c(\mbox{minimális Hamilton-kör)}    \\ 
 							   & = 2\cdot OPT 								  \\
\end{align*}

\subsubsection{Christofides algoritmusa}

\emph{egy $\frac{3}{2}$-approximációs algoritmus}. Ez a második lépést két részre bontja:

\begin{enumerate}[(a)]
\item vegyük a feszitőfa páratlan foku pontjait. Ebből $2m$ darab van, tehát
páros sok. $G$--ből elhagyjuk azon pontjait amelyek nincsenek benne ebben a
halmazban. Legyen $H$ az így kapott gráf.
\item keressünk egy minimális összsúlyú teljes párosítást $H$--ban. A párosítás
éleit egészítsük ki (adjuk hozzá) a feszitőfához. $F$ most egy $n-1+m$ élű gráf.
Ebben keressük az Euler kört.
\end{enumerate}

Tudjuk, hogy $F$ súlya $\leq$OPT. Tehát ahhoz, hogy az approximáció igaz legyen
a fenti procedúrával kapott párosítás legfeljebb $\frac{1}{2}$OPT értékű lehet.
Jelöljük a minimális Hamilton-kört $G$ gráfban $H$--val, $G'$ segédgráfban
$H'$--vel. 

$c(H') \leq$OPT, mert $H$--t levágásokkal át tudjuk alakítani úgy, hogy csak
$G'$--beli csúcsokat tartalmazzon, miközben nem nő az összköltsége, és az így
kapott $G'$--beli Hamilton-körnél $H'$ nyilván nem nagyobb súlyú.

A minimális összsúlyú teljes párosítás pedig legfeljebb $\frac{c(H')}{2}$
költségű, mert $H'$ páros élei és páratlan élei is teljes párosítást alkotnak, a
kettő költségének összege $c(H')$, tehát van legfeljebb $\frac{c(H')}{2}$
összköltségű párosítás, aminél nem nagyobb a minimális összsúlyú teljes
párosítás.

Mig a $2$--approximáció idő igénye $cn^2$, addig a Christofidesé $cn^3$.
Láthatjuk, hogy a jobb közelítés ára a nagyobb idő igény. A Hamilton--kör
további javításával a pontosság még tovább csiszolható, például a Lin--Kernighan
módszerrel $1-2$~\% is elérhető. 
