\newpage
\section{A lineáris és egészértékű programozás alkalmazása hálózati folyamproblémákra.}

\[
\begin{rcases} 
G(V,E) \mbox{ irányított gráf} \\
s,t \in V \mbox{ két kitűntet csúcs} \\
c : E \mapsto \mathbb{R}^+ \mbox{ nem negatív kapacitásfüggvény}\\
x : E \mapsto \mathbb{R}^+ \mbox{ tetszőleges függvény}\\
\rho_x(v) \mbox{ -- } v\mbox{--be belépő élek összege } x \mbox{ szerint} \\
\delta_x(v) \mbox{ -- } v\mbox{--ből kilépő élek összege } x \mbox{ szerint}
\end{rcases} \parbox[c]{6.2cm}{ 
\begin{itemize}
  \item $x$ függvény akkor \emph{folyam}, ha \\ $\forall~v~\in~V - \{s,t\}$--re: \\ $\rho_x(v)=\delta_x(v)$
  \item $x$ megengedett, ha $\forall e \in E$--re $x(e) \leq c(e)$
  \item A folyam értéke: \\ $\delta_x(s)-\rho_x(s)=\rho_x(t)-\delta_x(t)$
\end{itemize}
 }
\]

\emph{Tétel. Ford Fulkerson: A maximális folyam értéke megegyezik a minimális vágás értékével.}

A bizonyításhoz először figyeljük meg a kővetkező lemmát: ha $x \in E \mapsto
\mathbb{R}^+$ és $\forall v \in V - \{s,t\}$ esetén $ \delta_x(v) \leq \rho_x(v)$
és $\delta_x(s) \leq \rho_x(t)$ akkor $x$ egy folyam. 

Ez igaz mert, ha $\forall v \in V-\{s,t\}$ csúcshoz felveszünk egy új terminálba
mutató élet, amelyekhez hozzárendeljük az $x'(e)=\rho_x(v)-\delta_x(v) \geq 0$
egyenletlőtlenséget (a belépők többségbe vannak a kilépőkhöz képest). A többi élen
maradjon meg a korábbi értékek, $x(e)=x'(e)$. 

Az így konstruált $x$ folyamhoz tartozik $G'$ gráf.  Mivel ez is folyam igaz,
hogy $\delta_{x'}(t) = \rho_{x'}(t)$. Ugyanakkor $ \rho_{x'}(t) \geq \rho_x(t)$
és $\delta_{x'}=\delta_{x}s$. Mivel csak egyenlőség állhat, ezért
$\rho_{x'}(t)=\rho_{x}(t)$. Hogy ez igaz legyen $x$ folyam kell, hogy legyen.


\begin{wrapfigure}{L}{0.35\textwidth}
  \begin{center}
    \vspace{-1.3cm}
\begin{displaymath}
\underbrace{
\begin{array}{c|ccc|c|}
\cline{2-5}
   &   & & & \\
   &  B& & & \\
 s &   & & &-1 \\
 t &   & & & 1 \\ 
 \cline{2-5}
   &   & & & 0 \\
   & E & & & 0 \\
\cline{2-5}
\end{array}}_{B^*}
\underbrace{
\begin{array}{|c|}
\hline
x\\
\\
\hline
\mu\\
\hline
\end{array}}_{x^*}
\leq
\underbrace{
\begin{array}{|c|}
\hline
\\
0\\
\\
\hline
\\
c\\
\\
\hline
\end{array}}_{M}
\end{displaymath}
  \vspace{-1.3cm}
  \end{center}
\end{wrapfigure}

AZ LP felíráshoz először is egészítsük ki a gráfot egy $e^*=(t,s)$ pszeudó éllel, ez lesz 
később majd a folyamértéke, $\mu$. Az így kapott gráf illeszkedési mátrixa legyen $B^*$.
Minden $v \in V - \{s,t\}$ csúcshoz tartozik egy sor, amelyre teljesül a $b_vx\leq 0$ feltétel 
(azaz a belépő élek ősszege nem kisebb, mint a kilépőké, mert $\delta_x(v) - \rho_x(v) \leq 
0 \rightarrow \delta_x(v) \leq \rho_x(v)$). 

A $B^*x^* \leq 0$ rendszer alapján $\begin{rcases} \delta_x(s)-\mu \leq 0 \\
\mu - \rho_x(t) \leq 0 \end{rcases} \Rightarrow \delta_x(s) \leq \mu \leq
\rho_x(t)$. Az előző lemmából meg következik, hogy $\delta_x(s)=\mu=\rho_x(t)$, 
vagyis, hogy a folyam értéke $\mu$. 

A maximális folyam max$\{ (0, \cdots,0,1)x^* : B^*x^* \leq 0; x^* \geq 0;$ $x
\leq c \}$. Az utolsó feltételt is hozzávesszük a mátrixhoz, mint az $E$ egy
egységvektor és a hozzá tartozó $c$ rész a $M$ vektorban.

A feladat duálisa min$\{ y(0, 0, \cdots, 0, c) :yM \geq (0, 0, \cdots, 0, 1);$
$y \geq 0 \}$. Fejezzük ki $y$--t mint $\left( \pi\left(v\right) |
w\left(e\right)\right)$.
Ekkor: $\begin{cases} 
(1)~\pi(v) \geq 0 \mbox{ és } w(e) \geq 0, \\
(2)~\mbox{minden } e = (u,v) \mbox{ élre }  \pi(u)-w(e) \geq 0, \\
(3)~\pi(t)-\pi(s) \geq 1. \end{cases}$ 

A duális változói közül $\pi$ a csúcsokhoz (menyivel nőt a potenciál), $w$ az
élekhez rendelhető (menyibe kerül nekem a szállitás az él mentén). A duális
feladat célja a $m_{\text{DLP}}= \mbox{min} \left\{ \sum_{e\in E}^{}
w(e)c(e)\right\}$ alak minimalizálása. Állítjuk, hogy ez megegyezik a hálózati
folyam minimális vágásának értékével (legyen ez $m_{\text{C}}$).

Bármely adott $m_{\mbox{C}}$ vágáshoz könnyen készíthető olyan $\pi$ és $w$
amelyre az $m_{\text{C}}= \mbox{min} \left\{ \sum_{e\in E}^{} w(e)c(e)\right\}$
következik. Bizonyításként adunk egy módszert ehhez: legyen $S$ (tartalmazza a
forráspontot) és $T$ (tartalmazza a terminál csúcsot) diszjunkt halmazok, ekkor:
$\begin{cases}
v \in S &\Rightarrow \pi(v)=0, \\
v \in T &\Rightarrow \pi(v)=1, \\
e \in (S,T) \mbox{ él } &\Rightarrow w(e)=1, \\
\mbox{másképp} &\Rightarrow w(e)=0.
\end{cases}$

Erre teljesül a $m_{\text{C}}= \mbox{min} \left\{ \sum_{e\in E}^{}
w(e)c(e)\right\}$, amiből adódik az $m_{\mbox{DLP}} \leq m_{\mbox{C}}$, már csak
a másik irányú egyenlőtlenséget kell beállítani. Az $M$ mátrix totálisan
unimoduláris, tehát $y$ is egész értékű elemekből áll (mivel a duális
feladatban, min$\left\{ yb:yA=c; y\geq 0 \right\}$, szereplő $c$ is az).

Legyen adott $(\pi,w)$ optimális, egészértékű megoldás, ebből kiindulva
elkészítünk egy $(\pi',w')~0$ vagy $1$ értékű optimális megoldást. Definiáljuk a
következő függvényeket:
$\pi'(v)=
\begin{cases}
0, &\mbox{ha } \pi(v) \leq \pi(s), \\
1, &\mbox{egyébként}, 
\end{cases}$ és
$w'(e)=
\begin{cases}
0, &\mbox{ha } w(e)=0, \\
1, &\mbox{ha } w(e) \geq 1.
\end{cases}$ 

Ekkor $(\pi',w')$--re $(1)$ és $(3)$ teljesül. A $(2)$-öt indirekt bizonyítjuk.
Tegyük fel, hogy egy adott $e=(u,v)$ él esetén $\pi'(u)-\pi'(w)+w'(e)<0$, ekkor
a $0-1$ érétkűségük miatt $\pi'(u)=w'(e)=0$. Ez $\pi'(v)=1$ esetben valósulna
meg, amikor $\pi'$ definíniciója miatt $\pi(v) > \pi(s) \geq \pi(u)$, ami
ellentmondana $(2)$--nek, mert $w'$ definíciója miatt $w(e)=0$. A megoldás
optimális mert $\sum w'(e)c(e) \leq \sum w(e)c(e)$, mivel $w'(e) \leq w(e)$.

Most visszatérve az $S$ és $T$ halmazainkra legyen, $S=\{v \in V: \pi'(v)=0\} $
és $T=\{v \in V: \pi'(v)=1\}$. Egy adott $e=(u \in S, v \in T)$ élre $w'(e)=1$.
Minden más élen $w$ csak akkor lehet egy, ha $c(e)=0$, mert egyébként $w'(e)=0$
változtatása után a feltételek továbbra is fennállnának, de a $\sum w'(e)c(e)$
csökkenne. Ezért $S$ és $T$ között feszülő élek alkotta vágás értéke
$m_{\mbox{DLP}}$, így $m_{\mbox{c}} \leq m_{\mbox{DLP}}$.

Ezzel egy általános bizonyítást adtunk a Ford--Fulkerson tételre, amely így
átfogalmazott alakban általánosabb kérdések megválaszolására is alkalmazható.

\subsection{Minimális költségű folyam keresése}

Minden élhez rendeljünk egy k költséget, amely kifejezi, hogy egy egységnyi folyam
átviteli azon menyibe kerül. Ekkor kitűzhető egy olyan feladat, ami a legalább $M$
nagyságú folyamok között keres minimális költségűt. Lineáris programozás alakban:

\[ \mbox{min} \left\{kx: B^*x^* \leq 0, x^* \geq 0, x \leq c,mu \geq M \right\}. \]

Ha az élek kapacitásai egész értékűek, akkor egész értékű folyam is választható.
Ha az élek költsége is egészértékű, akkor ismert hatékony algoritmus megoldására,
Ford--Fulkersontól.

\subsection{Többtermékes folyam}

Adott egy $G=(V,E)$ irányított gráf és abban k darab pontpár:
$(s_i,t_i)_{i = \overline{1,k}}$. $G$--t továbbra is képzelhetjük út-- vagy
csőhálózatnak, az $(s_i, t_i)$ pontpárok pedig a $k$ darab szállítandó termék
termelő--, illetve fogyasztóhelyének felelnek meg. Végül legyen adott egy $c:E
\mapsto \mathbb{R}^+$ kapacitásfüggvény.

A feladat egy megoldása abból áll, hogy minden élhez $k$ darab számot fogunk
hozzárendelni, megmondva, hogy az egyes termékből ott éppen mennyi halad át.
Erre az esetre is alkalmazható a lineáris programozás, sőt a legalább két
termékes folyamoknál már az egyetlen ismert hatékony algoritmus ($k=1$--re a
javító utas algoritmus egész értékű megoldást ad polinomiális időben). Az egész
értékű feladat viszont már itt is NP nehéz (mivel a feladatott leíró mátrix nem
totálisan unimoduláris ekkor).
